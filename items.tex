\section{Potential items not covered in \gls{Operations}} \label{sec:items}

In the November 2023 DMvF2F several items were identified as not falling in \gls{DM} scope.
There are other items which have appeared on the summit, are useful, but have unclear owners in operations.
\gls{SITCOM} is also building more functionality with no clear owner/maintainer in operations.

This is a list of some of these items which a summit \gls{software} team might improve and maintain:

\subsection{ RubinTV }
 RubinTV  has been developed in \gls{SITCOM}, there have been no reviews and there is a sole maintainer.
This will certainly exist through operations and should be owned , reviewed and maintained y more than one person.
This is somewhat being looked at by FIG \citeds{SITCOMTN-059}.

\subsection{ Summit \gls{QA} tools}
\jira{SITCOM-700} is a missing functionality ticket in this area.

In DM a \gls{QA} working group was setup in 2018 \citedsp{LDM-622} which reported in 2019 \citeds{DMTN-085}.
DM dealt mainly with tooling and considered pipelines - some of that tooling may end up on the summit especially derivatives of the metrics framework \citeds{DMTN-098}.

In any case there will be pipeline code running on the summit producing QA outputs which a summit team should be able to configure/update/maintain.

There are also performance tools from RPF.

\subsection{ Rapid Analysis }
Sitting under QA tools and RubinTV is the Rapid Analysis Framework \citeds{SITCOMTN-100}.
DM would like eventually to consolidate this with Prompt Processing but for now it is on its own.
This is possible something best maintained by DM but deployed by the summit team.


\subsection{ \gls{OCPS} }
\gls{OCPS} \citedsp{DMTN-133}  was  written by \gls{DM} but is  not obviously a \gls{DM} maintenance job.
DM would like to merge this with Rapid Analysis and drop OCPS entirely.

\subsection{ OODS and s3daemon }
The s3daemon code \url{https://github.com/lsst-dm/s3daemon} amd OODS \url{https://github.com/lsst-dm/ctrl_oods}
transfer files from the summit.
Maintaining this code  falls in DM but installation ans keeping it running on the summit belongs with the summit team.

\subsection{ Reporting }

There are several types of reporting from operator logs to end of night reports.
A technote from Frossie is due on this.

\subsection{ \gls{Camera} Visualization}
Tony built a tool for previewing camera images - we believe is this going to be on the summit ?
There are camera diagnostic machines available to run it and other camera specific diagnostics.
There are Camera developers (or one at least) to be added to the Observatory team in operations to support this.


\subsection{ Header Service }
 \gls{DM} put this in place as it was  needed but it was not really \gls{DM} functionality.
There was discussion of camera providing this but camera now use this service to get the header.

\subsection{ Exposure Log}
This has been ill defined and worked on by many.
It is currently tied up in the Consolidated Database \citedsp{DMTN-258}

\subsection{ Live ObsCore}
Live ObsCore \citeds{DMTN-236} which exposes some of the butler DB to TAP is another publishing of summit information which may be needed at the summit.
The software would be DM middleware but keeping it running on summit and installing could be Summit Team.

\subsection{ Exposure Log  insertion at \gls{USDF}}
  This probably  should be \gls{USDF}

\subsection{ \gls{EFD}}
 EFD this is the biggest item provided by \gls{DM} - it should probably stay with SQuaRE but \gls{DM} was never sized for 24/7 support.

\subsection{ Data Wrangling }
Calibrations, and butler databases, schema migrations, keeping sites/test stands in sync are all
part of "data wrangling".
Though not part of this team this function must interact with the team.
